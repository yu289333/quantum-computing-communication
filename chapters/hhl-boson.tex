\section{HHL algorithm}
Harrow, Hassidim, and Lloyd (HHL) algorithm approximately solves linear equation $A x = b$ where $x$ and $b$ are $n$ dimensional vectors and $A$ is a $n x n$ hermitian matrix of complex numbers. The problem would be solved if we knew the orthonormal eigen-vectors ${v_i}, i=1, 2, ...n$ and the eigen-values ${\lambda_i}$ of $A$ because the eigen-values of $A^{-1}$ are  ${1/\lambda_i}$. By writing vector $b$ in terms of the eigen-vectors $b = {\sum^n}_{i=1} \beta_i v_i$, we can achieve our goal
\begin{equation}
    x = A^{-1} b = \sum^{n}_{i=1} \frac {\beta_i} {\lambda_i} v_i .
\end{equation}

Drawing from the technique used in the Deautsch's and Shor's algorithms, we need to "kick up" the eigen-values to the phases. Let's try
\begin{equation}\label{hhl-1}
    {\sum^n}_{i=1} \beta_i e^{i t \lambda_i} \keta{v_i}
\end{equation}
where $t$ is a constant to be determined soon. With the circuit of phase estimation, we then can bring the phases down to the second set of qubits
\begin{equation}\label{hhl-2}
    {\sum^n}_{i=1} \beta_i e^{i t \lambda_i} \keta{v_i} \keta{t\lambda_i}.
\end{equation}
We then apply controlled rotation gate to kick values of the second set of qubits to the $\theta$ of the third set of qubits, which contains only one, and have
\begin{equation}\label{hhl-3}
    \sum_{i=1}^n \beta_i e^{i t \lambda_i} \keta{v_i} \keta{t\lambda_i} \keta{cos\theta_i=\frac 1 {t\lambda_i}}.
\end{equation}

\subsection{Circuit diagram}
\begin{figure}[h]\label{HHL}
\begin{quantikz}%[slice all, slice style={shorten <=8mm}, slice label style = {yshift=-38mm} ]
    \lstick{\ket{0}} & \qwbundle{1} & \qw               & \qw       & \qw       & targ{RY}  & \meter{} &\cw \rstick{} \\
    \lstick{\ket{0}} & \qwbundle{n} &\gate{H^{\otimes n}} &\ctrl{1}     & \gate{IQFT} & \ctrl{-1} & \qw &\gate{QFT} &\ctrl{1}       &\qw \rstick{Bob's 1st bit} \\
    \lstick{\ket{b}} & \qwbundle{n} & \qw               & \targ{e^{itA}} & \qw       &\qw       &\qw    &\qw       &\targ{$e^{-itA}$} & \qw \rstick{{\ket{x}}}
\end{quantikz}
\caption{HHL algorithm}
\end{figure}

\subsection{Complexity}

\section{Boson sampling algorithms}

\section{Complexity}

\section{Noise, and error correction}
\section{Channel capacity}
According to Shannon theorem, under noise, the channel capacity is $C = 2B (1+SNR)$.
